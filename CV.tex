% LaTeX Curriculum Vitae Template
%
% Copyright (C) 2004-2009 Jason Blevins <jrblevin@sdf.lonestar.org>
% http://jblevins.org/projects/cv-template/
%
% You may use use this document as a template to create your own CV
% and you may redistribute the source code freely. No attribution is
% required in any resulting documents. I do ask that you please leave
% this notice and the above URL in the source code if you choose to
% redistribute this file.

\documentclass[letterpaper]{article}

\usepackage{hyperref}
\usepackage{geometry}

% Comment the following lines to use the default Computer Modern font
% instead of the Palatino font provided by the mathpazo package.
% Remove the 'osf' bit if you don't like the old style figures.
% \usepackage[T1]{fontenc}
% \usepackage[sc,osf]{mathpazo}

% Set your name here
\def\name{Peter B.~Denton}

% Replace this with a link to your CV if you like, or set it empty
% (as in \def\footerlink{}) to remove the link in the footer:
% \def\footerlink{http://www.owlnet.rice.edu/\textasciitilde pbd1/}
\def\footerlink{\null}

% The following metadata will show up in the PDF properties
\hypersetup{
colorlinks = true,
urlcolor = black,
pdfauthor = {\name},
pdfkeywords = {physics, mathematics},
pdftitle = {\name: Curriculum Vitae},
pdfsubject = {Curriculum Vitae},
pdfpagemode = UseNone
}

\geometry{
body={6.5in, 9.7in},
left=1.0in,
top=0.9in
}

% Customize page headers
\pagestyle{myheadings}
\markright{\name}
\thispagestyle{empty}

% Custom section fonts
\usepackage{sectsty}
\sectionfont{\rmfamily\mdseries\Large}
\subsectionfont{\rmfamily\mdseries\itshape\large}

% Other possible font commands include:
% \ttfamily for teletype,
% \sffamily for sans serif,
% \bfseries for bold,
% \scshape for small caps,
% \normalsize, \large, \Large, \LARGE sizes.

% Don't indent paragraphs.
\setlength\parindent{0em}

% Make lists without bullets
\renewenvironment{itemize}{
\begin{list}{}{
\setlength{\leftmargin}{1.5em}
}
}{
\end{list}
}

\begin{document}

% Place name at left
{\huge \name}

% Alternatively, print name centered and bold:
%\centerline{\huge \bf \name}

\vspace{0.1in}

\begin{minipage}{0.45\linewidth}
138 Village at Vanderbilt\\
Nashville TN 37212
\end{minipage}
\begin{minipage}{0.45\linewidth}
\begin{tabular}{ll}
Phone: & (616) 450-0749\\
Email: & \href{mailto:peterbd1@gmail.com}{\tt peterbd1@gmail.com} \\
\end{tabular}
\end{minipage}
\subsection*{Education}
\begin{itemize}
\item Ph.D. Physics, Vanderbilt University (in progress).
\item B.S. Physics, Rice University, 2010.
\item B.A. Mathematics, Rice University, 2010.
% \item Introduction to accelerator physics, United States Particle Accelerator School, 2009.
% \item Multi-variable calculus, differential equations, Grand Rapids Community College, 2005-2006.
\end{itemize}
\subsection*{Honors and Awards}
\begin{itemize}
\item Division of Particles \& Fields travel grant to the APS April 2013 meeting.
\item The Robert T.~Lagemann Award for highest academic achievement by a first-year graduate student.
\item ``Topping-up" McMinn Fellowship.
\end{itemize}
\subsection*{Employment}
\begin{itemize}
\item DOE funded research assistant with Thomas J.~Weiler at Vanderbilt University, Spring 2011--Present.
\item Teaching assistant at Vanderbilt University, Fall 2010--Fall 2012.
\item Research assistant with Pantelides at Vanderbilt University, Summer--Fall 2010.
\item Lee Teng Internship at Fermilab with Tanaji Sen, Summer 2009.
\item Teaching assistant for the Physics Department at Rice University, Fall 2009.
\item Bonner Labs at Rice University with Bill Llope, Summer 2008.
\item Writing consultant at Rice University, 2007--2010.
\end{itemize}
\subsection*{Selected Research Topics}
\begin{itemize}
\item Cosmic ray anisotropy at the highest energies as seen through Pierre Auger Observatory and JEM-EUSO.
\item Beyond the standard model contribution to integral dispersion integrals for $pp$ scattering at LHC energies.
\item The possibility of cosmic rays producing black holes in the atmosphere.
\item Matching Weinberg's Higgs portal to $N_{\rm{eff}}$ measurements.
\item Characterizing electroweak bremsstrahlung with effective operators to reduce spin suppression.
\end{itemize}
\subsection*{Language Skills}
\begin{itemize}
\item 
\LaTeX, beamer, python, matplotlib, C++, ROOT, MATHEMATICA, MATLAB, FORTRAN, gnuplot, java, html, and javascript.
\end{itemize}
\subsection*{Teaching}
\begin{itemize}
\item Teaching assistant for various freshman physics labs, 2010-Present.
\item Tutoring English, math, and physics at middle school, high school, undergraduate, and graduate school levels, 2005-Present.
\end{itemize}
\subsection*{Professional Societies}
\begin{itemize}
\item Member, American Physical Society
\end{itemize}
\end{document}
